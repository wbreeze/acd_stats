\documentclass[10pt,twocolumn]{article}

\title{Simulation study of power for goodness of fit to normal tests with
  rounded data}
\author{Douglas Lovell}
\date{Draft, January, 2020}

\usepackage{newclude}
\usepackage{url}
\usepackage{graphicx}
\graphicspath{{figures/}}

\begin{document}
\maketitle
\begin{abstract}
Where zero is total failure, and ten is perfect performance, the continuous
spectrum of performances between the two is rounded by the judges of
aerobatic contests to grades from zero to ten in half point increments.
This reports a simulation study, with such rounded data, of the power of
goodness of fit tests to normal.
\end{abstract}

\section{Procedure}

We run separate simulations for data set sizes $n$ from twelve to sixty.
We simulate true fits by generating three thousand $n$ size data sets
sampled from a normal distribution with mean of five and standard deviation
of 2.5.
We simulate false fits by generating one thousand $n$ size data sets
sampled from each of left skew, right skew, and uniform distributions.
On each generated data set we run the following goodness of fit tests:
\begin{itemize}
\end{itemize}

Having the measurements, we determine best cutoff points for each data size
and with each test. We define "best" in two ways. The first is minimizing type
one errors, the second minimizing type two errors.
We report the power properties-- number of
true and false positives, number of true and false negatives. We also train
a classfier using the results of all of the tests, and report the power
of that classifier.

\bibliographystyle{plain}
\bibliography{references}
\end{document}
