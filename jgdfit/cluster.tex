\section{Clustering}

In order to apply the Chi-Square metric, we need to cluster grades which
were given by the judge less than five times together with neighboring
grades.

Example

The optimal method for doing this is the one that has least impact on
the mean and variance of the data. The method described by
Greenacre \ref{Greenacre}
and implemented by the
greenclust \ref{greenclust}
R package provides one; however, it reorders the clusters.
We need to join only adjacent clusters, to maintain the order.

Another paper
\ref {partitioning}
from ten authors at NASA Ames and San Jose State University
maintains order of partitions, but does not enable the constraint of
minimum partition size. This and other algorithms for k-shape
ordered partitioning require a cumulative
function that measures the quality of each partition.
Here, we find that the mean and variance do not always decrease or
increase when joining two clusters.

The constraints at play here are:
\begin{enumerate}
\item{A minimum number of five grades in any cluster}
\item{Only adjacent clusters may be combined}
\item{Combine the least number}
\item{Have the weighted mean and variance close to the original}
\end{enumerate}

Where $n$ is the number of grade values in the range of grades,
we explore the $2^{n-1}$ combinations of joins using the heuristic
of choosing to first try joining clusters with the smallest number of grades,
and a bound of minimum count of joins. The partition
is complete when no cluster contains less than five grades. If two solutions
have the same number of joins, we select the one with minimum of
$(\mu' - \mu) + (\sigma' - \sigma)$, in which
$\mu' - \mu$ is
the difference in the weighted mean, and
$\sigma' - \sigma$ is
the difference in the weighted variance.

We generate a normal model from the clustered data set and report
the fit.
We report whether any grades were joined by clustering.
We report the mean and variance for both the clustered normal model
and the normal model generated by FPS.
