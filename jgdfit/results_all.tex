\section{Judge grade distributions for all figures}

We now look at distributions of all of the grades given by a judge for
a given flight-- all pilots and all figures. We look only at flights for
which there were more than five pilots and more than eleven grades given.

We also look only at flights in which all of the pilots flew the same figures.
To do so, we select for the flight format, taking only Known, Unknown, second
unknown, second and third known flights. The distribution of flight formats is
as in Table \ref{tab:allformat}:

\begin{table}[tbp]
  \centering
  \begin{tabular}{r r r r r r}
  Known & Unknown & Unknown II & Flight 1 \\
  1488 & 1160 & 27 & 19 \\
  \end{tabular}
  \caption{Distribution of flight formatsn for all-figures measurements}
  \label{tab:allformat}
\end{table}

In all, we look at 2,511 powered flights and 183 glider flights. The flights
breakdown by category is as in table \ref{tab:allcats}.

\begin{table*}[tbp]
  \begin{tabular}{r r r r r}
  primary  & sportsman & intermediate  &  advanced  & unlimited  \\
      177  &      1160 &          854  &       444  &        59  \\
  \end{tabular}
  \caption{Distribution of categories for all-figures measurements}
  \label{tab:allcats}
\end{table*}

Pilot count is the number of pilots who participated in a flight.
Grade count is the number of pilots times the number of figures, equals
the number of grades given by a judge for all of the figures and pilots in
a flight. Their distributions in this data set are as in
Table \ref{tab:allcounts}.

\begin{table*}[tbp]
  \begin{tabular}{r | r r r r r r}
  & Min. & 1st Qu. & Median & Mean & 3rd Qu. & Max. \\
  Pilot Count &  6 &  7 &  8 & 9.277 &  11 &  26 \\
  Grade Count & 35 & 65 & 81 & 94.55 & 110 & 359 \\
  \end{tabular}
  \caption{Distribution of pilot and grade counts for all-figures measurements}
  \label{tab:allcounts}
\end{table*}

The correlations between the four measures are strong, with the exception
of Lilifords to Shapioro-Wilk. Find their values in Table \ref{table:allcorr}.

\begin{table}[tbp]
  \begin{tabular}{l | r r r r}
  & sw & lf & ad & cvm \\
   \hline
  sw   & 1.000  & 0.432  & 0.792   & 0.616 \\
  lf   & 0.432  & 1.000  & 0.709   & 0.814 \\
  ad   & 0.793  & 0.709  & 1.000   & 0.937 \\
  cvm  & 0.616  & 0.814  & 0.937   & 1.000 \\
  \end{tabular}
  \caption{Correlation of all-figures measurements}
  \label{table:allcorr}
\end{table}

Table \ref{table:allsumm} shows summary results from the various goodness
of fit tests.
The numbers for measures found to be valid and invalid cover all of the
results.
The other two columns contain counts of p-value results from valid measures
greater than and less than or equal to the traditional cutoff of 0.05.

\begin{table}[tbp]
\centering
\begin{tabular}{l | r r r r}
& Invalid & Valid & $> 0.05$ & $<= 0.05$ \\
 \hline
chiSq.t.p   & 1149 & 1545 &  934 &  611 \\
chiSq.d.p   & 1149 & 1545 &  821 &  724 \\
sw.p.value  &    0 & 2694 &  106 & 2588 \\
lf.p.value  &    0 & 2694 &   36 & 2658 \\
ad.p.value  &    0 & 2694 &   32 & 2662 \\
cvm.p.value &    0 & 2694 &   47 & 2647 \\
\end{tabular}
\caption{All figure GOF measure p-value summary}
\label{table:allsumm}
\end{table}

The null hypothesis for these tests is that the distributions fit a normal
distribution. Using the traditional cutoff,
the Chi-Squared test finds no support for fit to normal in roughly half
of the cases.
The other tests almost never find support for fit to normal.

We can go deeper by looking at the distributions of p-values and then
looking at some specific examples within the various quantiles.

Table \ref{table:alldist} shows the spread of p-values from the various tests.
The numbers are from only those tests reported as valid.
The minimum values from the data are always zero, and so omitted from the table.

\begin{table*}[tbp]
  \begin{tabular}{l | r r r r r r}
  & 1st Qu. & Median &  Mean & 3rd Qu. &  Max. \\
   \hline
  chiSq.t.p & 0.006 & 0.126 & 0.272 & 0.491 & 0.999 \\
  chiSq.d.p & 0.002 & 0.067 & 0.216 & 0.358 & 0.993 \\
  sw.p.value & 0.012e-04 & 0.896e-04 & 0.785e-02 & 0.209e-02 & 0.452 \\
  lf.p.value & 0.0 & 9.200e-07 & 2.462e-03 & 1.121e-04 & 2.208e-01 \\
  ad.p.value & 2.000e-08 & 7.280e-06 & 2.612e-03 & 3.129e-04 & 1.912e-01 \\
  cvm.p.value & 4.400e-07 & 2.942e-05 & 3.327e-03 & 5.980e-04 & 2.781e-01 \\
  \end{tabular}
  \caption{All figure GOF measure p-value distributions}
  \label{table:alldist}
\end{table*}

Note that except for Chi-Squared, the mean value is greater than the
third quantile value,
demonstating that more than three-quarters of the values fall below the mean.

We can look at plots of the distributions to find more insights about the
fits.
One power study \cite{steele} finds that the Anderson-Darling test is
among three most powerful. It has high correspondence with the other
tests.
In the following, we show two plots side-by-side for representative p-values
from the
Anderson-Darling measure. The left plot is a histogram of the judge grades
with derived normal curve superimposed. The right plot is a standard Q-Q
plot using the derived normal curve.

At the smallest p-value seen in Figure \ref{fig:all:min} we find a skew
toward higher grades. The judge graded a majority of 9.0 with a mean grade
of 8.5. The scarce 4.0 will always be seen as an outlier with respect to
the normal curve. The upper tail of the normal curve exceeds the highest grade.

\begin{figure}[p]
  \centering
  \includegraphics[width=\columnwidth]{{all-Min-f11646-jX341-11.73-01}.png}
  \caption{All figures example at minimum Anderson-Darling p-value}
  \label{fig:all:min}
\end{figure}

At the first quantile upper limit seen in Figure \ref{fig:all:one}
we also find a skew toward the higher grades, but also a longer tail
to the lower grades.
The upper tail of the normal curve exceeds the highest grade.

\begin{figure}[p]
  \centering
  \includegraphics[width=\columnwidth]{{all-1st-f8930-jX49-15.33-01}.png}
  \caption{All figures example at first quantile Anderson-Darling p-value}
  \label{fig:all:one}
\end{figure}

At the second quantile upper limit seen in Figure \ref{fig:all:two}
we have a bimodal distribution in which the judge gives a large number of
7.0 and 9.0 with fewer grades given with values 7.5, 8.0, 8.5 and then
a lesser number given with values 5.0, 6.0, 6.5, 9.5, and 10.0.
That the upper tail of the normal curve exceeds the highest grade is
beginning to look like a pattern.

\begin{figure}[p]
  \centering
  \includegraphics[width=\columnwidth]{{all-Med-f9281-jX41-12.64-01}.png}
  \caption{All figures example at second quantile Anderson-Darling p-value}
  \label{fig:all:two}
\end{figure}

At the third quantile upper limit seen in Figure \ref{fig:all:three}
the picture has improved somewhat. There are a few too many grades with
value 6.0 and 8.0, too few with grades 7.5 and 10.0. To fit the curve,
the judge should have given a dash of grades with value 10.5, which
is not a valid grade.

\begin{figure}[p]
  \centering
  \includegraphics[width=\columnwidth]{{all-3rd-f6004-jX443-12.91-01}.png}
  \caption{All figures example at third quantile Anderson-Darling p-value}
  \label{fig:all:three}
\end{figure}

At the maximum value seen in Figure \ref{fig:all:max}
we have a judge who spreads their grades out more than in the other
examples. The spread does show a strong, though excessive peak at the
mean grade of 7.0. There are a few too many grades with value 4.0, 5.0,
8.0, and 9.5. There are too few with values 5.5, 7.5, and 8.5.
This judge almost manages to get the entire normal curve within the
range of grades; however, the upper tail still exceeds the maximum grade
of 10.0.

\begin{figure}[p]
  \centering
  \includegraphics[width=\columnwidth]{{all-Max-f11663-jX1033-15.00-01}.png}
  \caption{All figures example at maximum Anderson-Darling p-value}
  \label{fig:all:max}
\end{figure}

