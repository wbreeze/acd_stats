\section{Aerobatic Competition}

Aerobatics, as practiced by the World Air Sports Federation (FAI)
Aerobatics Comission (CIVA) and by the International Aerobatic Club (IAC),
is a judged sport. The pilot competitors perform in front of judges, who
give grades to the pilots. In this way, it is similar to figure skating,
gymnastics, and diving.

CIVA flies a handful of European and World contests each year.
Results are posted online \cite{civa-results} in human readable format.
The largest contests are the World Aerobatic Championship \cite{wac}
and the World Advanced Aerobatic Championship, which attract about sixty
pilots, mostly from Europe and the United States.

The IAC flies about forty regional contests and one national contest
per year in the United States.
Results from IAC contests are posted online \cite{iaccdb}
in human and machine readable formats.

\subsection{Grading}

A ``flight program'' consists of each of the competing pilots flying a sequence
of figures in front of the judges. The sequences of figures are predetermined
for the flight program by any one of several methods.
Judges receive the sequences that each pilot will fly in order to evaluate
each figure flown by the pilot and give it a grade.

The grade is a value from zero to ten in half point increments.
A flight program produces a grade $g_{j,p,f}$
from each judge $j$ for each figure $f$ flown by each pilot $p$.

\section{Tested hypothesis}

CIVA contests use a statistical system of converting judge grades to scores,
which system is documented in the FAI sporting code as FPS \cite{fps}.

The CIVA FPS represents subsets $G$ of the grades $g_{j,p,f}$
as the normal distribution derived from their mean and variance,
$Normal(\mu_G, \sigma^2_G)$.
It does this in order to normalize and
identify grades not in keeping with those from other judges.
The validity of the normal distribution
representation of the grades is fundamental to the correctness of the
method.  If it is not valid, it may be argued that the method is fundamentally
flawed.

This paper measures goodness of fit to the normal distribution of
grades given by judges in 2710 flight programs from IAC contests
during six years from 2014 to 2019, and reports the results.

For a particular judge $j$ we have a set
$G = g_{j,p_1,f_1}, g_{p_2,f_1}, ..., g_{j,p_n,f_1}, ..., g_{j,p_n,f_m}$
as grades given $n$ different pilots flying the same $m$ figures
in a flight program. The grades are discrete, measured in $1/2$
increments from 0 to 10.  We test the null hypothesis $H0$ that the grades
$G$ fit
the normal distribution derived from their mean and variance,
$Normal(\mu_G, \sigma^2_G)$.
