\section{Aerobatic Competition}

Aerobatics, as practiced by the World Air Sports Federation (FAI)
Aerobatics Comission (CIVA) and by the International Aerobatic Club (IAC),
is a judged sport. The pilot competitors perform in front of judges, who
give grades to the pilots. In this way, it is similar to figure skating,
gymnastics, and diving.

CIVA flies a handful of European and World contests each year.
Results are posted online \cite{civa-results} in human readable format.
The largest contests are the World Aerobatic Championship \cite{wac}
and the World Advanced Aerobatic Championship, which attract about sixty
pilots, mostly from Europe and the United States.

The airplanes flown at the world level cost as much as \$400,000 U.S. dollars.
The pilots as individuals and as part of national teams invest years of
training and practice whose value equals or exceeds that of the airplane.
The total investment brought by pilots and national teams to the
CIVA world contests can be estimated at about forty million U.S. dollars.

The IAC flies about forty regional contests and one national contest
per year in the United States.
Results from IAC contests are posted online \cite{iaccdb}
in human and machine readable formats.

In aerobatics, pilots fly figures made up of a basic flight path
overlaid with rolls, in which the airplane moves about it's fuselage
while it's center continues on the flight path.

A ``flight program'' consists of each of the competing pilots flying a sequence
of figures in front of the judges. The sequences of figures are predetermined
for the flight program by any one of several methods.
Judges receive the sequences that each pilot will fly in order to evaluate
the sequence actually flown against the sequence commited to fly by the pilot.

\subsection{Grading}

The judges give each figure a grade from zero to ten in half point increments.
They start from a perfect score of ten, and systematically deduct for flaws in
the direction of the flight path, radius of loops, and degrees of roll.  Thus,
a flight program produces a three-dimensional matrix of grades consisting of
Pilot X Judge X Figure.  Each pilot receives one grade from each judge for each
figure.

To complicate the grading slightly, a zero grade can take multiple forms.
One form is that incremental flaws resulted in deductions that summed to
ten or more. This is known as a ``soft zero''. A second form is that the
figure flown did not match the figure that was supposed to be flown.
It might have been missing a roll element or not have been flown along
the prescribed flight path. This is known as a ``hard zero''. CIVA uses
a form known as a ``presentation zero'' that a judge may use to indicate
suspicion of a hard zero. A ``conference zero'' indicates that the judge
changed their grade to a hard zero in review following the performance
of the pilot.

A judge may also give a grade of ``average''. A grade of average indicates
that the judge was either distracted or unsure of the figure that was
prescribed to be flown and therefore unable to evaluate the figure performed
against the figure prescribed. It is preferred that the judge give an
average under these conditions, rather than make-up a grade.
The effect of averages is a reduction in the number of judges evaluating
the figure.

\subsection{Scoring}

CIVA contests use a statistical system of converting judge grades to scores
documented in the FAI sporting code as FPS \cite{fps}.
The system normalizes judge grades using their mean and standard deviation,
then determines upper and lower bounds based upon the mean and standard
deviation of the grades from all of the judges. It then replaces
individual judge grades exceeding these bounds with fitted (mean, normalized)
grade values before computing a final score for each pilot.

The FPS makes the normalization and outlying grade computations using
grades from figure groups, which usually consist of grades for
one figure flown by all of the pilots in one flight program, and with all
of the grades from one flight program.

This paper examines whether the distribution of individual judge grades from
aerobatic contests fit the normal distribution as used by the CIVA FPS.
