\subsection{Shapiro-Wilk}

The grades given by the judges are rounded to increments
of 0.5 from a continuum of performances by the pilots.
In order to use the Shapiro-Wilk test, we
must perturb the grades in order to make them unique.

Two methods to perturb the grades are
\begin{itemize}
\item add values selected from a random uniform distribution between -2.5 and
2.5.
\item add values selected from a random normal distribution.
\end{itemize}

Applying a uniform distribution is the Smirnov transformation described
in \cite{lemkol}.
The method uniformly and
reliably perturbs the data within the range and has the advantage of
prior use in practice. A disadvantage is that the mean
of the perturbed values can end-up anywhere between -2.5 and 2.5 added to
the actual grade.

Using the random normal distribution ensures that the mean of the perturbed
values will remain close to the actual grade value. The disadvantage is that
we must choose a standard deviation for the random normal distribution
such that the resulting perturbations are
extremely rarely, with very low probability, outside of the range -2.5 to
2.5.  A second, potential disadvantage is that this is a variation of the
Smirnov transformation that, so far as we know,
we are inventing for this study

We perturbed the grades using random values chosen from a normal distribution
with mean equal to zero and standard deviation equal to $\sqrt{5/12n}$,
approximating the distribution of Bates \cite{Bates} between -2.5 and 2.5.
We report the p-value for the Shapiro-Wilk normality test after perturbing
the grades.
