\section{Conclusions}

The judge grade data is rounded data because judges are limited to giving
grades in increments of 0.5. The literature provides precedent for applying
random perturbations to make the data continuous, then using the continuous
tests for goodness of fit to normal.

The Chi-Squared test used with discrete data is shown in the literature
to be less powerful than the continuous tests applied to rounded data with
perterbations. However, in any case, although considerable effort was
applied, there are frequently too few data
values in these sets of grades with which to provide a valid Chi-Squared
result.

The goodness of fit measures looking at all of the grades given by a judge
during a flight fail to support the null hypothesis-- that the grades
fit a normal distribution derived from their mean and standard deviation
--more than half of the time.
The measures looking at FPS figure groups, that encompass fewer grades but
for a single or small number of figures, fail to support the null hypothesis
slightly less than half the time.

Looking at histograms of the grade frequencies overlaid with the normal
model curves reveals a large variation in grade distributions given by
judges and illustrates the lack of conformity to the normal distribution.

The lack of conformity of judge grade distributions to the normal distribution
suggests reconsideration of the FPS scoring method, that uses at its
foundation a normal distribution model of the judge grades.
