\section{FPS groups}

The distribution of grades given by a judge over all figures in a flight
provides measures of a larger number of grades. The FPS however uses figure
groups that divide the judges' grades into subsets grouped by same figure
flown.  We now look at distributions of grades given by a judge within the FPS
groups for a given flight.

The distribution of flight formats is as in Table
\ref{tab:fpsformat}, also selecting for formats in which all of the pilots
fly the same figures.

\begin{table}[tbp]
  \centering
  \begin{tabular}{r r r r r r}
  Known & Unknown & Unknown II & Flight 1 \\
  7989 & 5734 & 175 & 49 \\
  \end{tabular}
  \caption{Distribution of flight formats for FPS group measurements}
  \label{tab:fpsformat}
\end{table}

In all, we look at 13,298 powered figure groups and 649 glider. The figure
groups breakdown by category is as in table \ref{tab:fpscats}.

\begin{table*}[tbp]
  \begin{tabular}{r r r r r}
  primary  & sportsman & intermediate  &  advanced  & unlimited  \\
      531  &      6345 &         4439  &      2313  &       319  \\
  \end{tabular}
  \caption{Distribution of categories for FPS group measurements}
  \label{tab:fpscats}
\end{table*}

The distributions of pilot and grade count in this data set are as in
Table \ref{tab:fpscounts}.

\begin{table*}[tbp]
  \begin{tabular}{r | r r r r r r}
  & Min. & 1st Qu. & Median & Mean & 3rd Qu. & Max. \\
  Pilot Count &  6 &  7 &  9 & 10.82 &  13 &  26 \\
  Grade Count & 12 & 13 & 15 & 16.20 & 18 & 48 \\
  \end{tabular}
  \caption{Distribution of pilot and grade counts for FPS group measurements}
  \label{tab:fpscounts}
\end{table*}

The correlations between the four measures are stronger than for the
all-figures data.
Find their values in Table \ref{table:fpscorr}.

\begin{table}[tbp]
  \begin{tabular}{l | r r r r}
  & sw & lf & ad & cvm \\
   \hline
  sw   & 1.000  & 0.711  & 0.897   & 0.813 \\
  lf   & 0.711  & 1.000  & 0.870   & 0.926 \\
  ad   & 0.897  & 0.870  & 1.000   & 0.975 \\
  cvm  & 0.813  & 0.926  & 0.975   & 1.000 \\
  \end{tabular}
  \caption{Correlation of FPS figure group measurements}
  \label{table:fpscorr}
\end{table}

Table \ref{table:fpssumm} shows summary results from the various goodness
of fit tests.
The numbers for measures found to be valid and invalid cover all of the
results.
The other two columns contain counts of p-value results from valid measures
greater than and less than or equal to the traditional cutoff of 0.05.
There was only one valid result for the Chi-Squared test due to the
reduced number of grade data points within the FPS groups. For that reason,
we omit the Chi-Squared result.

\begin{table}[tbp]
\centering
\begin{tabular}{l | r r r r}
& Invalid & Valid & $> 0.05$ & $<= 0.05$ \\
 \hline
sw.p.value  & 0 & 13947 & 8311 & 5636 \\
lf.p.value  & 6 & 13941 & 6351 & 7590 \\
ad.p.value  & 6 & 13941 & 6458 & 7483 \\
cvm.p.value & 6 & 13941 & 6640 & 7301 \\
\end{tabular}
\caption{FPS groups GOF measure p-value summary}
\label{table:fpssumm}
\end{table}

The tests find support for fit to normal, using the traditional cutoff of
0.05, in a little more than half, roughly 55\% of the cases.

Table \ref{table:alldist} shows the spread of p-values from the various tests.
The numbers are from only those tests reported as valid.
The minimum values from the data are always zero, and so omitted from the table.

\begin{table*}[tbp]
  \begin{tabular}{l | r r r r r r}
  & 1st Qu. & Median &  Mean & 3rd Qu. &  Max. \\
   \hline
  sw.p.value & 0.165e-01 & 0.854e-01 & 0.180 & 0.267 & 0.999 \\
  lf.p.value & 0.435e-02 & 0.371e-01 & 0.115 & 0.153 & 0.997 \\
  ad.p.value & 0.610e-02 & 0.407e-01 & 0.107 & 0.144 & 0.953 \\
  cvm.p.value & 0.696e-02 & 0.446e-01 & 0.109 & 0.152 & 0.925 \\
  \end{tabular}
  \caption{FPS group GOF measure p-value distributions}
  \label{table:fpsdist}
\end{table*}

At the smallest p-value seen in Figure \ref{fig:fps:min} we find a
judge who, looking at three figures from seven pilots, almost always
gave a grade of 9.0 with the few exceptions being a grade of 8.0.

\begin{figure}[p]
  \centering
  \includegraphics[width=\columnwidth]{{fps-Min-f6142-jX1442-6.67-01}.png}
  \caption{FPS groups example at minimum Anderson-Darling p-value}
  \label{fig:fps:min}
\end{figure}

At the first quantile upper limit seen in Figure \ref{fig:fps:one}
we find a judge who most often gives a grade of seven, rarely lower and
sometimes higher. This results in a skew toward the lower values.

\begin{figure}[p]
  \centering
  \includegraphics[width=\columnwidth]{{fps-1st-f6432-jX157-22.00-01}.png}
  \caption{FPS groups example at first quantile Anderson-Darling p-value}
  \label{fig:fps:one}
\end{figure}

At the second quantile upper limit we can look at distributions
from three judges looking at the same figure group.
This is instructive for seeing the further variety of distributions
of grades that judges generate.

In Figure \ref{fig:fps:twoa} we have simply a judge
who favors the grade of 8.0 for thirteen pilots all flying the same
figure. The normal distribution would have more of those eights as
7.5 or 7.0 grades, resulting in a little bit of a skew toward the higher
grades, mostly due to too many in the middle.

\begin{figure}[p]
  \centering
  \includegraphics[width=\columnwidth]{{fps-Med-f10531-jX216-10.00-01}.png}
  \caption{FPS groups first example at second quantile Anderson-Darling p-value}
  \label{fig:fps:twoa}
\end{figure}

In figure \ref{fig:fps:twob} the judge gives an almost uniform distribution
from 7.0 to 9.0, although with a complete absence of grade 7.5.

\begin{figure}[p]
  \centering
  \includegraphics[width=\columnwidth]{{fps-Med-f10531-jX216-10.00-02}.png}
  \caption{FPS groups second example at second quantile Anderson-Darling p-value}
  \label{fig:fps:twob}
\end{figure}

In figure \ref{fig:fps:twoc} the judge uses a much larger range of grades,
from 4.0 to 9.0.
Due to there being only thirteen grades in all, the distribution contains
zero, one, two, or three instances of each possible grade in the range.

\begin{figure}[p]
  \centering
  \includegraphics[width=\columnwidth]{{fps-Med-f10531-jX216-10.00-03}.png}
  \caption{FPS groups third example at second quantile Anderson-Darling p-value}
  \label{fig:fps:twoc}
\end{figure}

At the third quantile upper limit seen in Figure \ref{fig:fps:three}
the judge gives most of their grades in the middle of their range, with
a couple of instances above or below. More in the middle and fewer at the
tails makes for a better fit to normal, but few of the individual judge
grade counts match the density that would be expected from the normal.

\begin{figure}[p]
  \centering
  \includegraphics[width=\columnwidth]{{fps-3rd-f10767-jX125-17.20-01}.png}
  \caption{FPS groups example at third quantile Anderson-Darling p-value}
  \label{fig:fps:three}
\end{figure}

At the maximum value seen in Figure \ref{fig:fps:max}
we have a better filled spread of grades with, as for the third quantile,
more grades near the mean.

\begin{figure}[p]
  \centering
  \includegraphics[width=\columnwidth]{{fps-Max-f7758-jX1695-32.50-01}.png}
  \caption{FPS groups example at maximum Anderson-Darling p-value}
  \label{fig:fps:max}
\end{figure}
